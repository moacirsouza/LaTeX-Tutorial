\documentclass{article}

\title{My first Document}
\date{08/12/2025}
\author{Moka}

\usepackage{amsmath}
\usepackage{graphicx}
\usepackage{subcaption}

\begin{document}

  \pagenumbering{gobble}
  \maketitle

  \newpage
  \pagenumbering{arabic}
  \tableofcontents

  \newpage
  \section{Section: Text}

    Hello World! This is a Section. But what is a Section in the first place? 
    And, honestly, how long should a Section be, before being split into two or
    more lines? Or does it even matter? If I explicilty split my text before or
    after column 80, like in the old days, will it make a difference? I don't
    think so ;)

    Now I have a secong paragraph that's not explicitly marked as a paragraph.
    How will LaTeX deal with this text? Will it put it in a new line and start
    the text with a left shift? (Yes, it will!)

    And how about this one? Will it left shift and automatically justify it, 
    just like last time? (Also yes!)

    \subsection{Subsection}

      Structuring a document is easy!

    \subsubsection{Subsubsection}

      More text (and some math).

      \begin{equation*}
        f(x) = x^2
      \end{equation*}
      
      Even more text, with the embeded formula $f(y) = y^2 - y +1$, as an example.

      \paragraph{Paragraph}

        Some more text.

        \subparagraph{Subparagraph}

          Even more text
  
  \newpage
  \section{Another Section: Equations}

    \subsection{Not aligned}
      \begin{equation*}
        1 + 2 = 3
      \end{equation*}

      \begin{equation*}
        1 = 3 - 1
      \end{equation*}

    \subsection{Aligned (by the equal sign)}
      \begin{align*}
        1 + 2 &= 3 \\
        1 &= 3 - 2
      \end{align*}

    \subsection{Several common functions}
      \begin{align*}
        f(x) &= x^2 \\
        g(x) &= \frac{1}{x} \\
        F(x) &= \int^a_b \frac{1}{3}x^3 \\
        G(x) &= \frac{1}{\sqrt{x^2}} \\
        H(x) &= \left(\frac{1}{\sqrt{x}}\sin(\theta)\right)
      \end{align*}

    \subsection{Matrices}

    Matrices only work if inside "math environments", like "equation" or a pair
    of dollar signs.

      \begin{equation*}
        \left[
          \begin{matrix}
            1 & 0\\
            0 & 1
          \end{matrix}
        \right]
      \end{equation*}

  \newpage
  \section{Figures}

    A bunch of Figures :)

    \begin{figure}[h!]
      \centering
      \begin{subfigure}[b]{0.3\linewidth}
        \includegraphics[width=\linewidth]{gato_preto.jpg}
        \caption{A baby black cat.}
        \label{fig:blackcat}
      \end{subfigure}
      \begin{subfigure}[b]{0.3\linewidth}
        \includegraphics[width=\linewidth]{gato_branco.jpg}
        \caption{A baby white cat.}
        \label{fig:whitecat}
      \end{subfigure}
      \begin{subfigure}[b]{0.31\linewidth}
        \includegraphics[width=\linewidth]{gato_laranja.jpg}
        \caption{A baby orange cat.}
        \label{fig:orangecat}
      \end{subfigure}
      \caption{Three baby cats, side by side}
      \label{fig:cats}
    \end{figure}

    Figure \ref{fig:cats} shows three baby cats, right next to each other.
    There is a black cat is on the left (Figure \ref{fig:blackcat}), a white
    one in the middle (Figure \ref{fig:whitecat}) and an orange one on the
    right (Figure \ref{fig:orangecat}).

\end{document}
